\chapter{基础知识回顾}

\section{多变量微积分复习}
有$n$个自变量,$(x_1,x_2,\ldots,x_n)$的函数$f$的微分为
	\begin{equation}
		df = \sum_{i=1}^n \frac{\partial f}{\partial x_i} = \frac{\partial f}{\partial x_1} + \frac{\partial f}{\partial x_2} + \ldots + \frac{\partial f}{\partial x_n}
	\end{equation}
	其中偏导数
	\begin{equation}
		\frac{\partial f}{\partial x_i} = \lim_{h \to 0} \frac{f(x_1,x_2,\ldots,x_i + h, \ldots, x_n) - f(x_1,x_2,\ldots,x_i, \ldots, x_n)}{h}
	\end{equation}
	在本文中同时采用下列记法表示一阶和二阶偏导数,
	\begin{equation}
		\frac{\partial f}{\partial x_i} \equiv \partial_{x_i} f \equiv f_{x_i}, \hspace{0.25in}\frac{\partial^2 f}{\partial x_i x_j} \equiv \partial_{x_i x_j}^2 f \equiv f_{x_i x_j}
	\end{equation}
	用下面的记号表示向量,
	\begin{equation}
		\vec{u} \equiv \mathbf{u}
	\end{equation}
	向量微分算子. 在三维笛卡尔坐标系${\mathbf{i},\mathbf{j}, \mathbf{k} }$,考虑$f(x,y,z):\mathbb{R}^3\to \mathbb{R}$和$[u_x(x,y,z),u_y(x,y,z),u_z(x,y,z)]:\mathbb{R}^3\to \mathbb{R}^3$ \\
	梯度:
		\begin{equation}
			\nabla f = \partial_x f \mathbf{i}  + \partial_y f \mathbf{j} + \partial_z f \mathbf{k}
		\end{equation}
	散度:
	\begin{equation}
		\nabla \mathbf{u} = \partial_x u_x + \partial_y u_y + \partial_z u_z
	\end{equation}
	旋度:
	\begin{equation}
		\nabla \times \mathbf{u} = (\partial_y u_z - \partial_z u_y )\mathbf{i} + (\partial_z u_x - \partial_x u_z)\mathbf{j} + ( \partial_x u_y - \partial_y u_x )\mathbf{k}
	\end{equation}
	拉普拉斯算子:
	\begin{equation}
		\Delta f \equiv \nabla^2 f = \partial_x^2 f + \partial_y^2 f + \partial_z^2 f
	\end{equation}
	向量的拉普拉斯算子:
	\begin{equation}
	\Delta \mathbf{u} \equiv \nabla^2 \mathbf{u} = \nabla^2 \mathbf{u}_x\mathbf{i}  + \nabla^2 \mathbf{u}_y\mathbf{j} + \nabla^2 \mathbf{u}_z\mathbf{k}
	\end{equation}
	在其他坐标系下(圆柱坐标系,球坐标系),以上算子具有不同的形式.
\section{偏微分方程定义}
热传导方程
	\begin{equation}
		\frac{\partial T}{\partial t} = \frac{\partial^2 T}{\partial x^2} + \frac{\partial^2 T}{\partial y^2} + \frac{\partial^2 T}{\partial z^2}
	\end{equation}

	\begin{mydef}
		偏微分方程\\
		具有两个或以上自变量的函数$u$的变化由其偏导数确定,有时也由其自身以及自变量确定.
		\\
		当未知函数大于1个时,得到偏微分方程组.
	\end{mydef}
	\begin{equation}
		F(x_1,x_2,\ldots,x_n,u,\partial_{x_1} u, \ldots, \partial_{x_n} u, \partial_{x_1}^2 u, \partial_{x_1 x_2}^2,\ldots, \partial_{x_1 x_2 \ldots x_n}^n u)
	\end{equation}

	\begin{mydef}
		偏微分方程的阶数\\
		偏微分方程的阶数指方程中最高阶偏导数的阶数.
	\end{mydef}

	
一阶方程的一般定义,定义域为$(x,y)$平面的某区域$D$.
	\begin{equation}
		F(x,y,u,u_x,u_y) = 0
	\end{equation}
二阶方程
	\begin{equation}
		G(x,y,u,u_x,u_y,u_{xx},u_{yy},u_{xy}) = 0
	\end{equation}
	
	偏微分方程一般有无穷多个解,为了获得唯一解,需要补充辅助条件,通常包括边界条件和初始条件.
	\begin{mydef}
		边界和边界条件\\
		边界:偏微分方程通常需要在一定的空间区域内求解.当所关心的区域是受限的有限区域或半无限区域时,边界的处理非常重要. 如何区分边界点,内点和外点?以二维问题为例.考虑任意的二维域$\Omega$.如果以$p$点为圆心作圆,无论半径多小,圆总会包含属于$\Omega$中点和不属于$\Omega$的点,则$p$属于边界点.如果以$Q$点为圆心,可以画出一个圆只包含$\Omega$中的点,则$Q$为内点. 以$R$点为圆心,可以画出一个圆不包含$\Omega$中的任何点,则$R$为内点.$\Omega$的所有边界点的集合称为边界,记为$\partial \Omega$.三维的定义可以直接推广.
		在求解偏微分方程时,解除了要满足偏微分方程,同时在边界上需满足一定的边界条件,称为边界条件.\\
		三类基本的边界条件
		\begin{enumerate}[(1)]
			\item 第一类边界条件,或称为Dirichlet条件\\
				在边界上给定函数的值,比如
				\begin{equation*}
					u(x,y,z,t)=f(x,y,z,t), (x,y,z,t)\in \partial \Omega, t>0
				\end{equation*}
			\item 第二类边界条件,或称为Neumann 边界条件\\
				在边界上给定函数法向导数的值.
				\begin{equation*}
				\partial_n u(x,y,z,t)=f(x,y,z,t), (x,y,z,t)\in \partial \Omega, t>0
				\end{equation*}
			\item 第三类边界条件,或称为Robin条件\\
			 在边界上函数和函数的法向导数满足一定的关系
			 \begin{equation*}
			 	\alpha(x,y,z) \partial_n u(x,y,z,t) + u(x,y,z,t) =f(x,y,z,t), (x,y,z,t)\in \partial \Omega, t>0
			 \end{equation*}
			 
		\end{enumerate}
	除了这三类最常见的边界条件外,还有其他的边界条件,如混合边界条件,在边界的不同部分应用不同的边界条件类型.
	\end{mydef}

	\begin{mydef}
		初始条件.\\
		如果位置函数还是时间$t$的函数,还需要指定初始条件,在初始时刻$(t=0)$,需要给定整个区域内函数的值.
	\end{mydef}
	\begin{mydef}
		线性和非线性偏微分方程\\
		区分方程是线性或者是非线性,对于方程的求解十分重要.
		关于可微函数$u(x_1,x_2,\ldots,x_n)$的偏微分方程.
		 线性PDE\emph{(linear)}\\
		方程中不存在关于未知函数$u$及其偏导数的非线性项,同时边界条件和初始条件也必须是线性的.不是线性的方程都是非线性的.非线性方程又可以细分为一下几类:
		\begin{enumerate}[(1)]
			\item 半线性\emph{semilinear}\\
			最高阶偏导数前的系数仅是自变量的函数.
			\item 拟线性\emph{(quasilinear)}\\
			$m$阶偏导数前的系数可以是自变量,函数本身以及小于$m$阶偏导数的函数.
			\item 完全非线性\emph{(nonlinear)}\\
			不属于以上任意一类的方程.
		\end{enumerate}
	\end{mydef}	
	
	\begin{mydef}
		方程可以包含一个仅是自变量的函数f.如果$f=0$,则方程是齐次的,否则是非齐次的.
	\end{mydef}

	\begin{myexample}
		线性对流方程
		\begin{equation}
			u_t = a_x u_x + a_y u_y + a_z u_z
		\end{equation}
		其中对流速度$\vec{a}=a_x \vec{i} + a_y \vec{j} + a_z \vec{k}$为常数.
	\end{myexample}

	\begin{myexample}
		二维空间的Laplace方程
		\begin{equation}
			\Delta_2 u = 0
		\end{equation}
		其中
		\[\Delta_2 = \frac{\partial^2}{\partial x^2} + \frac{\partial^2}{\partial y^2}\]
		Laplace方程是齐次线性方程.
	\end{myexample}
		
	\begin{myexample}
		三维Possion方程
		\begin{equation}
			\Delta_3 u = -f(x,y,z)
		\end{equation}
		其中
		\[\Delta_3 = \frac{\partial^2}{\partial x^2} + \frac{\partial^2}{\partial y^2} + \frac{\partial^2}{\partial z^2}\]	
		Possion方程是非齐次线性方程.	
	\end{myexample}

	\begin{myexample}
		三维波动方程
		\begin{equation}
			u_tt = c^2 \Delta_3 u
		\end{equation}
		其中$c$是波传播速度.
	\end{myexample}

	\begin{myexample}
		一维梁振动方程
		\begin{equation}
			u_{tt} + u_{xxxx} = F
		\end{equation}
		四阶非齐次线性方程.
	\end{myexample}

	\begin{myexample}
		无粘Bergers方程
		\begin{equation}
			u_t + uu_x = 0
		\end{equation}
		拟线性一阶齐次方程.
	\end{myexample}

	\begin{myexample}
		粘性Bergers方程
		\begin{equation}
			u_t + uu_x = \varepsilon u_{xx}
		\end{equation}
		二阶拟线性方程.
	\end{myexample}
	
	\begin{myexample}
		Kdv方程
		\begin{equation}
			u_t + uu_x + u_{xxx} = 0
		\end{equation}
		三阶拟线性方程.
	\end{myexample}

	\begin{myexample}
		反应扩散方程
		\begin{equation}
			u_t = u_{xx} + u(1-u) 
		\end{equation}
		二阶半线性方程.
	\end{myexample}

	\begin{myexample}
		真空中的Maxwell方程组
		\begin{subequations}
			\begin{align}
			\frac{1}{c}\mathbf{E}_t = \nabla \times \mathbf{H} \\
			\frac{1}{c}\mathbf{H}_t = -\nabla \times \mathbf{E} \\
			\nabla \mathbf{E} = \nabla \mathbf{H} = 0
			\end{align}
		\end{subequations}
	一阶线性方程组.
	\end{myexample}

	\begin{myexample}
		一维Euler方程.
		\begin{equation}
			\begin{bmatrix}
				\rho \\
				\rho u \\
				\rho e
			\end{bmatrix}_t + 
			\begin{bmatrix}
				\rho u \\
				\rho u^2 + p \\
				(\rho e + p)u
			\end{bmatrix}_x = 0
		\end{equation}
		一阶拟线性方程组
	\end{myexample}

	\begin{myexample}
		最小表面方程.
		\begin{equation}
			(1+u_y^2)u_{xx} - 2u_xu_yu_{xy} + (1+u_x^2)u_{yy} = 0
		\end{equation}
		该方程描述了由线圈支撑的肥皂泡的表面形状.二阶拟线性方程.
	\end{myexample}

	\begin{mydef}
		偏微分方程的存在性,唯一性和稳定性. \\
		当我们求解偏微分方程时,首先需要确定解是否存在?解如果存在,是否存在唯一解?如果初始条件或边界条件发生连续的变化,所得到的解是否也发生连续的变化?即微小的初始条件或边界条件改变,解的变化也应该是微小的.这是解的稳定性.
		给定初值和边值条件的偏微分方程称为定解问题、如果解存在,唯一且稳定,则称定解问题适定(well-posed);否则则称为不适定(ill-posed).
	\end{mydef}
	可以通过常微分方程的例子来理解上述概念.
	\begin{myexample}
		\begin{equation*}
			\frac{du}{dt}=u, \quad u(0)=1
		\end{equation*}
		解为:$u(t)=e^t, \quad 0\le t < \infty $.
	\end{myexample}

	\begin{myexample}
		\begin{equation*}
			\frac{du}{dt}=u^2, \quad u(0)=1
		\end{equation*}	
		解为: $u(t)=1/(1-t), \quad 0\le t < 1 $.	
	\end{myexample}

	\begin{myexample}
		\begin{equation*}
			\frac{du}{dt}=\sqrt u, \quad u(0)=0
		\end{equation*}	
		具有两个解: $u\equiv 0$ 和 $u=t^2/4$.
	\end{myexample}	

	\begin{myexample}
		Possion方程的解的唯一性.
		\begin{equation*}
			\Delta u = F \quad in \hspace{0.05in} \Omega 
		\end{equation*}
		在边界上$u = 0, on \ \partial \Omega $
		假定$u_1$和$u_2$满足方程和边界条件.考虑$\omega = u_1 - u_2$. 则在$\Omega$内$\Delta \omega = 0$ 且在边界上 $\omega = 0$.
		
		\begin{eqnarray*}
			\int_{\partial \Omega} \omega \nabla \omega \cdot \mathbf{n} dS & = & \int_{\Omega} \nabla \cdot (\omega \nabla \omega) dV \\
			& = & \int_{\Omega} [(\nabla \omega)^2 + \omega \Delta \omega] dV
		\end{eqnarray*}
	则要求$\nabla \omega=0$,即$\omega = const.$.根据边界条件,$\omega = 0$,即$u_1=u_2$,解是唯一的.
	\end{myexample}
	
\section{微分算子和叠加原理}
\textit{k}阶方程的解必须是\textit{k}次可微的.我们定义$C^k(D)$为在域$D$内所有\textit{k}次连续可微的函数集.特别的,定义$D$上的连续函数集为$C^0(D)$或$C(D)$.满足\textit{k}阶PDE方程的$C^k$中的函数,称为PDE的经典解(classic solution)或强解(strong solution).需要指出的是有时会出现不存在经典解的情形,此时的解称为弱解(weak solution).\\
不同函数集之间的映射称为算子(\textit{operators}).算子$L$对函数$u$的操作用$L(u)$表示.本书主要考虑描述函数导数的算子称为微分算子(differential operators).\\
如果算子满足下列关系
\begin{equation}
	L[a_1u_1+a_2u_2]=a_1L[u_1] + a_2L[u_2]
\end{equation}
其中$a_1,a_2$为任意常数,$u_1,u_2$为任意函数,则称算子为线性算子(linear operator).线性微分方程自然定义了线性算子,方程可以表示为$L[u]=f$,其中$L$为线性算子,$f$是给定函数.\\
线性算子在整个数学中起着中心作用,在PDE中更是如此.$u_i, 1\le i \le n $满足线性微分方程$L[u_i]=f_i$,则线性组合$\upsilon:=\sum_{i=1}^n \alpha_i u_i$满足方程$L[\upsilon]=\sum_{i=1}^{n} \alpha_i f_i$. 特别地,当$u_1,u_2,\cdots, u_n$满足齐次方程$L[u_i]=0$时,这些函数的线性组合也满足方程,这称为叠加原理(superposition principle).

\section{二阶偏微分方程的分类}
	二阶线性偏微分方程
	\begin{equation} \label{2nd order linear pde}
		L(u) = a u_{xx} + 2bu_{xy} + c u_{yy} + du_x + eu_y +fu = g
	\end{equation}
	其中各项系数数是$x,y$的函数.\\
	包含二阶项的算子
	\begin{equation}
		L_0(u) = a u_{xx} + 2bu_{xy} + c u_{yy}
	\end{equation}
	称为算子$L$的主部,决定了方程的性质.可以通过判别式
	\begin{equation}
		\delta(L)(x,y) = b(x,y)^2 - a(x,y)c(x,y)
	\end{equation}
	的符号来判断方程的性质.
	
	\begin{mydef} 
		二阶线性偏微分方程的分类
		\begin{enumerate}[(1)]
		\item 如果在$\Omega$中的任意一点$(x,y)$,$\delta(L)(x,y)>0$,则方程是双曲型的(hyperbolic);
		\item 如果在$\Omega$中的任意一点$(x,y)$,$\delta(L)(x,y)=0$,则方程是抛物型的(parabolic);
		\item 如果在$\Omega$中的任意一点$(x,y)$,$\delta(L)(x,y)<0$,则方程是椭圆型的(elliptic);
		\end{enumerate}
	\end{mydef}

	\begin{mydef}
		坐标变换\\
		变换$(\xi,\eta)=(\xi(x,y),\eta(x,y))$是坐标变换,如果雅克比$J:=\xi_x\eta_y - \xi_y\eta_x$在$\Omega$中所有点都不为$0$.
	\end{mydef}

	\begin{mylemma}
		两变量线性二阶偏微分方程经过坐标变换后,方程类型不变.这表明,方程的类型是方程的内在属性,与坐标系无关.
	\end{mylemma}
	\begin{proof}
		令$(\xi,\eta)=(\xi(x,y),\eta(x,y))$是非奇异变换,方程的解可以表示为$\omega(\xi,\eta)=u(x(\xi,\eta),y(\xi,\eta))$,根据链式法则,
		\begin{eqnarray}
			u_x = \omega_{\xi}\xi_x + \omega_{\eta}\eta_x = \left(\xi_x \frac{\partial}{\partial \xi} + \eta_x \frac{\partial}{\partial \eta}\right) \omega \\
			u_y = \omega_{\xi}\xi_y + \omega_{\eta}\eta_y	= \left(\xi_y \frac{\partial}{\partial \xi}  + \eta_y \frac{\partial}{\partial \eta}\right)	\omega		
		\end{eqnarray}
		二阶导数项稍微复杂一点.
		\begin{eqnarray}
			\frac{\partial^2}{\partial x^2} & = & \left(\xi_x \frac{\partial}{\partial \xi} + \eta_x \frac{\partial}{\partial \eta}\right)^2  \nonumber  \\
			& = & \left[ 
			\xi_x \frac{\partial}{\partial \xi}\left(\xi_x  \frac{\partial}{\partial \xi}\right) + \xi_x \frac{\partial}{\partial \xi}\left(\eta_x  \frac{\partial}{\partial \eta}\right)
			\right]  +  	
			\left[ 
			\eta_x \frac{\partial}{\partial \eta}\left(\xi_x  \frac{\partial}{\partial \xi}\right) + \eta_x \frac{\partial}{\partial \eta}\left(\eta_x  \frac{\partial}{\partial \eta}\right)
			\right]  \nonumber \\
			& = & 
			\xi_{x}^2 \frac{\partial^2}{\partial \xi^2} + \eta_{x}^2 \frac{\partial^2}{\partial \eta^2} + 2\xi_x \eta_x \frac{\partial^2}{\partial \xi \partial \eta} \nonumber \\
			& + &			
			\left[ \frac{\partial}{\partial \xi}\left(\frac{\partial \xi}{\partial x}\right) \frac{\partial \xi}{\partial x} + \frac{\partial}{\partial \eta}\left(\frac{\partial \xi}{\partial x}\right) \frac{\partial \eta}{\partial x}\right] \frac{\partial}{\partial \xi} +
			\left[ \frac{\partial}{\partial \xi}\left(\frac{\partial \eta}{\partial x}\right) \frac{\partial \xi}{\partial x} + \frac{\partial}{\partial \eta}\left(\frac{\partial \eta}{\partial x}\right) \frac{\partial \eta}{\partial x}\right] \frac{\partial}{\partial \eta} \nonumber \\
			& = &
			\xi_{x}^2 \frac{\partial^2}{\partial \xi^2} + \eta_{x}^2 \frac{\partial^2}{\partial \eta^2} + 2\xi_x \eta_x \frac{\partial^2}{\partial \xi \partial \eta} + \frac{\partial^2 \xi}{\partial x^2} \frac{\partial}{\partial \xi} + \frac{\partial^2 \eta}{\partial x^2} \frac{\partial}{\partial \eta}
		\end{eqnarray}
		因此
		\begin{equation}
			u_{xx}=\xi_x^2 \omega_{\xi\xi} + \eta_x^2 \omega_{\eta\eta} + 2\xi_x\eta_x\omega_{\xi\eta} + \xi_{xx}\omega_{\xi} + \eta_{xx}\omega_{\eta}
		\end{equation}
		其他二阶导数项为
		\begin{eqnarray}
			u_{yy} & = & \xi_y^2 \omega_{\xi\xi} + \eta_y^2 \omega_{\eta\eta} + 2\xi_y\eta_y\omega_{\xi\eta} + \xi_{yy}\omega_{\xi} + \eta_{yy}\omega_{\eta} \\
			u_{xy} & = & \omega_{\xi\xi}\xi_x\xi_y + \omega_{\xi\eta}(\xi_x\eta_y+\xi_y\eta_x) + \omega_{\eta\eta}\eta_x\eta_y + \omega_{\xi}\xi_{xy} + \omega_{\eta}\eta_{xy}
		\end{eqnarray}
		将坐标变换后的偏导数代入到(\ref{2nd order linear pde}),
		\begin{equation}
			l[\omega]:=A\omega_{\xi\xi}+2B\omega_{\xi\eta} + C\omega_{\eta\eta} + D\omega_{\xi} + E\omega_{\eta} + F\omega = G
		\end{equation}
		其中
		\begin{eqnarray}
			A & = & a\xi_x^2 + 2b\xi_x\xi_y + c\xi_y^2 \\
			B & = & a\xi_x\eta_x + b(\xi_x\eta_y+\xi_y\eta_x) + c\xi_y \eta_y \\
			C & = & a\eta_x^2 + 2b\eta_x\eta_y + c\eta_y^2
		\end{eqnarray}
		方程的性质决定于主部,根据简单的计算,系数满足关系式
		\begin{equation}
			\begin{bmatrix}
				A & B \\
				B & C
			\end{bmatrix} = 
			\begin{bmatrix}
				\xi_x & \xi_y \\
				\eta_x & \eta_y
			\end{bmatrix} 
			\begin{bmatrix}
				a & b \\
				b & c 
			\end{bmatrix}
			\begin{bmatrix}
				\xi_x & \eta_x \\
				\xi_y & \eta_y
			\end{bmatrix}
		\end{equation}
	上式两边取行列式,
	\begin{equation}
		-\delta(l)=AC - B^2 = J^2(ac - b^2) = -\delta(L).
	\end{equation}
	因此非奇异变换不改变方程的性质.
	
	\end{proof}
通常我们可以在一个坐标系下将方程表示为标准形式(canonical form),
\begin{mydef}
	双曲型方程的标准形式
	\begin{equation}
		l[\omega]=\omega_{\xi\eta}+l_1[\omega] = G(\xi,\eta)
	\end{equation}
	其中$l_1$是一阶线性微分算子.
\end{mydef}
\begin{mydef}
	抛物型方程的标准形式
	\begin{equation}
		l[\omega]=\omega_{\xi\xi}+l_1[\omega] = G(\xi,\eta)
	\end{equation}
\end{mydef}

\begin{mydef}
	椭圆型方程的标准形式
	\begin{equation}
	l[\omega]=\omega_{\xi\xi}+\omega_{\eta\eta}+l_1[\omega] = G(\xi,\eta)
	\end{equation}
\end{mydef}
\section{特征线法}
为了描述特征线法(method of characteristics),首先考虑简单的一维线性对流问题.
\begin{subequations}
	\begin{align}
	u_t + au_x  =  0  \label{linear-advection}\\
	u(x,0)  =  F(x)
	\end{align}
\end{subequations}
其中$u(x,t)$是$(x,t)$的未知函数,$a$是对流速度,$F(x)$是初值.特征线法是将偏微分方程转化为一系列的常微分方程求解.根据微分链式法则,
\begin{equation} \label{full-diff}
	\frac{du}{dt}=\frac{\partial u}{\partial t} + \frac{\partial u}{\partial x} \frac{dx}{dt}
\end{equation}
如果令
\begin{equation} \label{characteristic eq}
	\frac{dx}{dt} = a
\end{equation}
代入到(\ref{full-diff}),并与(\ref{linear-advection})比较,可得
\begin{equation} \label{transport-eq}
	\frac{du}{dt} = 0
\end{equation}
方程(\ref{characteristic eq})的解为一组曲线,沿着曲线,$u$的值将不变.这组曲线称为方程(\ref{linear-advection})的特征线(characteristic curves).求解偏微分方程(\ref{linear-advection})转化为求解方程(\ref{transport-eq})和(\ref{characteristic eq}).积分(\ref{characteristic eq})得到特征曲线族
\begin{equation}
	x(t) = at + \xi
\end{equation}
由于$a$为常数,所以特征线为直线.根据(\ref{transport-eq}),
\begin{equation}
	u(x,t) = u(\xi,0)=F(\xi)
\end{equation}
由于$\xi=x-at$,故
\begin{equation}
	u(x,t)=F(x-at).
\end{equation}
如果$F(x)$是$C^1$,可以验证$u(x,t)$满足偏微分方程(\ref{linear-advection})和初始条件.

\subsection{拟线性偏微分方程的特征线法}
特征线法被用来求解双曲型偏微分方程.本节主要介绍特征线法在一阶拟线性方程求解中的应用.考虑一般形式的一阶拟线性方程
\begin{equation} \label{quasi-linear-eq}
	a(x,y,u)u_x + b(x,y,u)u_y=c(x,y,u) \hspace{0.25in} in \hspace{0.1in} \Omega
\end{equation}
系数$a,b,c$是$(x,y,u)$的$C^1$函数.方程(\ref{quasi-linear-eq})可能的解$u(x,y)$在$\mathbb{R}^3$中构成了一个曲面$S$(例如,$u(x,y)=x^2-y^2$).$S$通常称为方程的积分面.解曲面$S$可以隐式表示为$f(x,y,u)=0$.在本例中$f(x,y,u)=u(x,y)-u$.根据隐式方程可以计算曲面的法向方向,即
\begin{equation}
	\nabla f = (u_x,u_y,-1)
\end{equation}
因此,向量$(u_x,u_y,-1)$是$S$上任一点(x,y,u)处的法向方向.根据观察,方程(\ref{quasi-linear-eq})
可以表示为
\begin{equation}
	(a,b,c)\cdot (u_x,u_y,-1) = 0
\end{equation}
上式表明,$(a,b,c)$与$(u_x,u_y,-1)$正交.向量$(a,b,c)$位于$S$上一点$(x,y,u)$的切平面内.由这这一条件可以得到Lagrange-Charpit方程
\begin{equation}
	\frac{dx}{a}=\frac{dy}{b}=\frac{du}{c}
\end{equation}

\begin{myexample}
	求解方程
	\begin{equation}
		u_t + uu_x = 0
	\end{equation}
%	Cauchy条件为$u(x,0)=2x,1\le x \le 2$.\\
	解: \\
	$a = 1, b = u, c = 0$,根据Lagrange-Charpit方程\\
	\begin{equation}
		\frac{dx}{u} = \frac{dt}{1} = \frac{du}{0}
	\end{equation}
	显然在特征线上$du=0$,即沿着特征线,$u$为常数.根据前两个方程可以得到
	\begin{equation}
		u dt= dx
	\end{equation}
	即特征线方程为$u-xt=c$.方程的解为
	\begin{equation}
		u = g(x-ut).
	\end{equation}
	
	
	
\end{myexample}
	例:偏微分方程 \cite{apde}
	\begin{equation*}
		u_x + 2xu_y = 2xu, 
	\end{equation*}
	对给定的3种条件分别进行求解.
	\begin{enumerate}
		\item $u(x,0)=x^2$
		\item $u(0,y)=y^2$
		\item 当$x \le 0$时, $u(x,0) = x^2$,当$y \le 0$时, $u(0,y) = y^2$
	\end{enumerate}
	